\documentclass[10pt,a4paper]{article}
\usepackage[right=0.5cm, left=0.5cm,top=0.5cm,bottom=0.5cm]{geometry}
\usepackage{enumitem}
\usepackage{forest}
\usepackage{amsfonts}
\usepackage{tikz}
\usetikzlibrary{positioning}
\usepackage{graphicx}
\usepackage{array, tasks}
\usepackage{blindtext}
\usepackage{fontspec}
\usepackage{amsmath,amsfonts,amssymb,mathrsfs,amsthm}
\usepackage{fancyhdr}
\usepackage{xcolor}
\usepackage{booktabs}
\usepackage[font={bf}]{caption}
% \captionsetup[table]{box=colorbox,boxcolor=orange!20}
\usepackage{float}
\usepackage{esvect}
\usepackage{tabularx}
\usepackage{pifont}
\usepackage{colortbl}
 \usepackage{fancybox}
 \mathversion{bold}
 \usepackage{pgfplots}
 % \usepackage[utf8]{inputenc}
\usepackage{tikz}
 \usepackage[tikz]{bclogo}%
 \usepackage{mathpazo}
\usepackage{ulem}
\usepackage{yagusylo}
\usepackage{textcomp}\usepackage{blindtext}
\usepackage{multicol}
\usepackage{varwidth}
\usetikzlibrary{calc,intersections}
\usepackage{pgfplots}
%\usepackage{fourier}
\pgfplotsset{compat=1.11}
\usepackage{tkz-tab}
\usepackage{xcolor}
\usepackage{color}
\usetikzlibrary{calc}
\mathchardef\times="2202
\usepackage[most]{tcolorbox}
\definecolor{lightgray}{gray}{0.9}
\definecolor{ocre}{RGB}{0,244,244} 
\definecolor{head}{RGB}{255,211,204}
\definecolor{browndark}{RGB}{105,79,56}
%\RequirePackage[framemethod=default]{mdframed}
\usepackage{tikz}
\usetikzlibrary{calc,patterns,decorations.pathmorphing,arrows.meta,decorations.markings}
\usetikzlibrary{arrows.meta}
\makeatletter
\tcbuselibrary{skins,breakable,xparse}
\tcbset{%
  save height/.code={%
    \tcbset{breakable}%
    \providecommand{#1}{2cm}%
    \def\tcb@split@start{%
      \tcb@breakat@init%
      \tcb@comp@h@page%
      \def\tcb@ch{%
        \tcbset{height=\tcb@h@page}%
        \tcbdimto#1{#1+\tcb@h@page-\tcb@natheight}%
        \immediate\write\@auxout{\string\gdef\string#1{#1}}%
        \tcb@ch%
      }%
      \tcb@drawcolorbox@standalone%
    }%
  }%
}
\newcommand{\Lim}{\displaystyle\lim}
\makeatother
\newcommand{\oij}{$\left( \text{O};\vv{i},\vv{j} , \vv{k}\right)$}
\colorlet{darkred}{red!30!black}
\newcommand{\red}[1]{\textcolor{darkred}{ #1}}
\newcommand{\rr}{\mathbb{R}}
\renewcommand{\baselinestretch}{1.2}
 \setlength{\arrayrulewidth}{1.25pt}
\usepackage{titlesec}
\usepackage{titletoc}
\usepackage{minitoc}
\usepackage{ulem}
%--------------------------------------------------------------

\usetikzlibrary{decorations.pathmorphing}
\tcbuselibrary{skins}

%%%%%%%%%%%
%-------------------------------------------------------------------------
\tcbset{
        enhanced,
        colback=white,
        boxrule=0.1pt,
        colframe=brown!10,
        fonttitle=\bfseries
       }
\definecolor{problemblue}{RGB}{100,134,158}
\definecolor{idiomsgreen}{RGB}{0,162,0}
\definecolor{exercisebgblue}{RGB}{192,232,252}
\definecolor{darkbrown}{rgb}{0.4, 0.26, 0.13}

\newcommand*{\arraycolor}[1]{\protect\leavevmode\color{#1}}
\newcolumntype{A}{>{\columncolor{blue!50!white}}c}
\newcolumntype{B}{>{\columncolor{LightGoldenrod}}c}
\newcolumntype{C}{>{\columncolor{FireBrick!50}}c}
\newcolumntype{D}{>{\columncolor{Gray!42}}c}

\newcounter{mysection}
\newcounter{mysubsection}
\newcommand{\mysection}[1]{%
    \stepcounter{mysection} % Increment the counter
    \textcolor{red}{\LARGE\themysection. #1 :}
}
\newcommand{\mysubsection}[2]{
    \stepcounter{mysubsection}
    \textcolor{red}{\large \themysection.#1. #2 :}
}
% \textcolor{red}{\LARGE\bfseries 1. Les équation du deuxiéme degrée :}

%------------------------------------------------------
\newtcolorbox[auto counter]{Definition}{enhanced,
before skip=2mm,after skip=2mm,
colback=yellow!20!white,colframe=lime,boxrule=0.2mm,
attach boxed title to top left =
    {xshift=0.6cm,yshift*=1mm-\tcboxedtitleheight},
    varwidth boxed title*=-3cm,
    boxed title style={frame code={
                        \path[fill=lime]
                            ([yshift=-1mm,xshift=-1mm]frame.north west)  
                            arc[start angle=0,end angle=180,radius=1mm]
                            ([yshift=-1mm,xshift=1mm]frame.north east)
                            arc[start angle=180,end angle=0,radius=1mm];
                        \path[left color=lime,right color = lime,
                            middle color = lime]
                            ([xshift=-2mm]frame.north west) -- ([xshift=2mm]frame.north east)
                            [rounded corners=1mm]-- ([xshift=1mm,yshift=-1mm]frame.north east) 
                            -- (frame.south east) -- (frame.south west)
                            -- ([xshift=-1mm,yshift=-1mm]frame.north west)
                            [sharp corners]-- cycle;
                            },interior engine=empty,
                    },
fonttitle=\bfseries\sffamily,
title={Definition ~\thetcbcounter}}
%------------------------------------------------------
\newtcolorbox[auto counter]{Proposition}{enhanced,
before skip=2mm,after skip=2mm,
colback=yellow!20!white,colframe=blue,boxrule=0.2mm,
attach boxed title to top left =
    {xshift=0.6cm,yshift*=1mm-\tcboxedtitleheight},
    varwidth boxed title*=-3cm,
    boxed title style={frame code={
                        \path[fill=blue]
                            ([yshift=-1mm,xshift=-1mm]frame.north west)  
                            arc[start angle=0,end angle=180,radius=1mm]
                            ([yshift=-1mm,xshift=1mm]frame.north east)
                            arc[start angle=180,end angle=0,radius=1mm];
                        \path[left color=blue,right color = blue,
                            middle color = blue]
                            ([xshift=-2mm]frame.north west) -- ([xshift=2mm]frame.north east)
                            [rounded corners=1mm]-- ([xshift=1mm,yshift=-1mm]frame.north east) 
                            -- (frame.south east) -- (frame.south west)
                            -- ([xshift=-1mm,yshift=-1mm]frame.north west)
                            [sharp corners]-- cycle;
                            },interior engine=empty,
                    },
fonttitle=\bfseries\sffamily,
title={Proposition ~\thetcbcounter}}
%------------------------------------------------------
\newtcolorbox[auto counter]{Thm}[1]{enhanced,
before skip=2mm,after skip=2mm,
colback=yellow!20!white,colframe=red,boxrule=0.2mm,
attach boxed title to top left =
    {xshift=0.6cm,yshift*=1mm-\tcboxedtitleheight},
    varwidth boxed title*=-3cm,
    boxed title style={frame code={
                        \path[fill=red]
                            ([yshift=-1mm,xshift=-1mm]frame.north west)  
                            arc[start angle=0,end angle=180,radius=1mm]
                            ([yshift=-1mm,xshift=1mm]frame.north east)
                            arc[start angle=180,end angle=0,radius=1mm];
                        \path[left color=red,right color = red,
                            middle color = red]
                            ([xshift=-2mm]frame.north west) -- ([xshift=2mm]frame.north east)
                            [rounded corners=1mm]-- ([xshift=1mm,yshift=-1mm]frame.north east) 
                            -- (frame.south east) -- (frame.south west)
                            -- ([xshift=-1mm,yshift=-1mm]frame.north west)
                            [sharp corners]-- cycle;
                            },interior engine=empty,
                    },
fonttitle=\bfseries\sffamily,
title={#1 ~\thetcbcounter}}
%------------------------------------------------------
\newtcolorbox[auto counter]{exemple}{
  % breakable,
  enhanced,
  colback=white,
  boxrule=0pt,
  arc=0pt,
  outer arc=0pt,
  title=Exemple ~\thetcbcounter,
  fonttitle=\bfseries\sffamily\large\strut,
  coltitle=problemblue,
  colbacktitle=problemblue,
  title style={
left color=exercisebgblue,
    right color=white,
    middle color=exercisebgblue  
  },
  overlay={
    \draw[line width=1pt,problemblue] (frame.south west) -- (frame.south east);
    \draw[line width=1pt,problemblue] (frame.north west) -- (frame.north east);
    \draw[line width=1pt,problemblue] (frame.south west) -- (frame.north west);
    \draw[line width=1pt,problemblue] (frame.south east) -- (frame.north east);
  }
}
%----------------------------------------------------
\newtcolorbox[auto counter]{application}{
  % breakable,
  enhanced,
  colback=white,
  boxrule=0pt,
  arc=0pt,
  outer arc=0pt,
  title=Application ~\thetcbcounter,
  fonttitle=\bfseries\sffamily\large\strut,
  coltitle=problemblue,
  colbacktitle=problemblue,
  title style={
left color=exercisebgblue,
    right color=white,
    middle color=exercisebgblue  
  },
  overlay={
    \draw[line width=1pt,problemblue] (frame.south west) -- (frame.south east);
    \draw[line width=1pt,problemblue] (frame.north west) -- (frame.north east);
    \draw[line width=1pt,problemblue] (frame.south west) -- (frame.north west);
    \draw[line width=1pt,problemblue] (frame.south east) -- (frame.north east);
  }
}
%----------------------------------------------------
\newtcolorbox[auto counter]{Activite}{
  % breakable,
  enhanced,
  colback=white,
  boxrule=0pt,
  arc=0pt,
  outer arc=0pt,
  title=Activité ~\thetcbcounter,
  fonttitle=\bfseries\sffamily\large\strut,
  coltitle=problemblue,
  colbacktitle=problemblue,
  title style={
left color=yellow!50!white,
    right color=white,
    middle color=yellow!20!white  
  },
  overlay={
    \draw[line width=1pt,problemblue] (frame.south west) -- (frame.south east);
    \draw[line width=1pt,problemblue] (frame.north west) -- (frame.north east);
    \draw[line width=1pt,problemblue] (frame.south west) -- (frame.north west);
    \draw[line width=1pt,problemblue] (frame.south east) -- (frame.north east);
  }
}
%---------------------------------------------
\newtcolorbox{mybox}[2]{enhanced,breakable,
    before skip=2mm,after skip=2mm,
    colback=white,colframe=#2!30!blue,boxrule=0.3mm,rightrule=0.3mm,
    attach boxed title to top center={xshift=0cm,yshift*=1mm-\tcboxedtitleheight},
    varwidth boxed title*=-3cm,
    boxed title style={frame code={
    \path[fill=#2!30!black]
    ([yshift=-1mm,xshift=-1mm]frame.north west)
    arc[start angle=0,end angle=180,radius=1mm]
    ([yshift=-1mm,xshift=1mm]frame.north east)
    arc[start angle=180,end angle=0,radius=1mm];
    \path[draw=black,line width=1pt,left color=#2!1!white,right color=#2!1!blue!65,
    middle color=#2!1!green]
    ([xshift=-2mm]frame.north west) -- ([xshift=2mm]frame.north east)
    [rounded corners=1mm]-- ([xshift=1mm,yshift=-1mm]frame.north east)
    -- (frame.south east) -- (frame.south west)
    -- ([xshift=-1mm,yshift=-1mm]frame.north west)
    [sharp corners]-- cycle;
    },interior engine=empty,
    },
title=#1,coltitle=black,fonttitle=\sffamily}
%---------------------------------------------
\newtcolorbox{boxone}{%
    enhanced,
    colback=brown!10,
    boxrule=0pt,
    sharp corners,
    drop lifted shadow,
    frame hidden,
    fontupper=\bfseries,
    notitle,
    overlay={%
        \draw[Circle-Circle, brown!70!black, line width=2pt](frame.north west)--(frame.south west); 
        \draw[Circle-Circle, brown!70!black, line width=2pt](frame.north east)--(frame.south east);}
    }

    \tikzset{ midd/.style={
decoration={
markings,
mark=at position 0.5 with {%
\makebox[4pt][r]{\arrow{triangle 45}},
},
},
postaction={decorate},{Circle[]}-{Circle[]}
},}%

\def\suite{(u_n)_{n\geq n_0}}
    
\begin{document}

\begin{tcolorbox}[title=\textcolor{blue}{\shadowbox{ Prof : Othmane Laksoumi}}
\hfill
\textcolor{blue}{\shadowbox{ Les suites numériques }}]

\end{tcolorbox}

\begin{mybox}{Lycée Qualifiant Zitoun}{gray}
    \begin{minipage}{8cm}
    \textcolor{darkbrown}{Année scolaire : } 2024-2025 \\
    \textcolor{darkbrown}{Niveau : } 2 Bac Sciences Physiques \\
    \textcolor{darkbrown}{Durée totale : } $10h$
    \end{minipage}
\end{mybox}

\begin{boxone}
{\Large\ding{45}}
\textcolor{red}{\large Contenus du programme :}
\begin{itemize}
   \item Limites des suites numériques de référence : $(n)_{n\geq 0} \ , \ (n^2)_{n\geq 0} \ , \ (n^3)_{n\geq 0} \ , \ (\sqrt{n})_{n\geq 0}\ , \ (n^p)_{n\geq 0}$ où $p$ est un entier naturel.
   \item Limites des suites numériques de référence : $\left(\displaystyle\frac{1}{n}\right)_{n\geq 0} \ , \ \left(\displaystyle\frac{1}{n^2}\right)_{n\geq 0} \ , \ \left(\displaystyle\frac{1}{\sqrt{n}}\right)_{n\geq 0} \ , \ \left(\displaystyle\frac{1}{n^p}\right)_{n\geq 0}$ où $p$ est un entier naturel.
   \item Suite convergente.
   \item Critères de convergence; convergence d'une suite croissante et majorée; convergente d'une suite décroissante et minorée.
   \item Suite divergente.
   \item Opérations sur les limites de suites; limittes et ordre.
\end{itemize}

{\Large\ding{45}}
\textcolor{red}{\large Les capacités attendues :}
\begin{itemize}
    \item Utiliser les suites arithmétiques et les suites géométriques pour étudier des exemples de suites de la forme :\\
     $u_{n+1} = au_n + b$ et $u_{n+1} = \displaystyle\frac{au_n+b}{cu_n + d}$
    \item Utiliser les suites de référence et les critéres de convergence pour déterminer les limites de suites numériques.
    \item Utiliser les suites numériques pour résoudre des problèmes variés de différentes domaines.
    \item Déterminer la limite d'une suite convergente $(u_n)$ de la forme : $u_{n + 1} = f(u_n)$ où $f$ est une fonction continue sur un intervalle $I$ et $f(I)\subset I$.
\end{itemize}

{\Large\ding{45}}
\textcolor{red}{\large Recommandations pédagogiques :} 
  \begin{itemize}
    \item Toute étude théorique de la notion de limite d'une suite est hors programme.
    \item En tenant compte du fait qu'une suite numérique est une fonction numérique sur l'ensemble des entiers naturels et à partir de limites de fonctions de référence on admettra dans une première étape les limites des suites : $(n)_{n\geq 0} \ , \ (n^2)_{n\geq 0} \ , \ (n^3)_{n\geq 0} \ , \ (\sqrt{n})_{n\geq 0}\ \text{et} \ (n^p)_{n\geq 0}$ et les suites : $\left(\displaystyle\frac{1}{n}\right)_{n\geq 0} \ , \ \left(\displaystyle\frac{1}{n^2}\right)_{n\geq 0} \ , \ \left(\displaystyle\frac{1}{\sqrt{n}}\right)_{n\geq 0} \ \text{et} \ \left(\displaystyle\frac{1}{n^p}\right)_{n\geq 0}$ où $p$ est un entier naturel supérieur ou égal à 3, quand $n$ tend vers $+\infty$.
  \end{itemize}
\end{boxone}


\begin{tabular}{|>{\raggedright\arraybackslash}p{17cm}|>{\centering\arraybackslash}p{0.8cm}|}
\hline
\rowcolor{head}

\centering Contenu du cour &
 Durée \\
\hline
\def\suite{(u_n)_{n\geq n_0}}

\vspace{1mm}
\mysection{Généralités sur les fonctions numériques}
\begin{Definition}
     \begin{itemize}
         \item Une suite numérique est une fonction définie sur $\mathbb{N}$ ou une partie de $\mathbb{N}$.
         \item Soit $f$ une fonction définie sur $\mathbb{N}$, alors $f$ est une suite numérique. Pour tout, $n\in\mathbb{N}$ on note $f(n)$ par $u_n$ et on note $f$ par $(u_n)_{n\geq 0}$ ou $(u_n)$.
     \end{itemize}
\end{Definition}

\begin{exemple}
    $(n^2 + n + 1)_{n\geq 0} \ , \ (\sqrt{n - 2})_{n\geq 2} \ , \ \left(\displaystyle\frac{1}{n}\right)_{n\geq 1} $ sont des suites numériques.
\end{exemple}
\mysubsection{1}{Suite majorée - suite minorée - suite bornée}

Soit $(u_n)_{n\geq n_0}$ une suite numérique.
\begin{Definition}
    \begin{itemize}
        \item On dit que la suite $(u_n)_{n\geq n_0}$ est majorée s'il existe un réel $M$ tel que : $(\forall n\geq n_0)\ \ u_n\leq M$.
        \item On dit que la suite $(u_n)_{n\geq n_0}$ est minorée s'il existe un réel $m$ tel que : $(\forall n\geq n_0)\ \ u_n\geq m$.
        \item On dit que la suite $(u_n)_{n\geq n_0}$ est bornée si elle est à fois majorée et minorée.
    \end{itemize}
\end{Definition}
\begin{exemple}
    \begin{enumerate}
        \item La suite $\left(\displaystyle\frac{1}{n} + 1\right)_{n\geq 1}$ est une suite majorée.
        \item La suite $(n^2 + 1)_{n\geq 0}$ est une suite minorée.
    \end{enumerate}
\end{exemple}

\mysubsection{2}{Monotonie d'une suite numérique}
\begin{Definition}
    \begin{itemize}
        \item On dit que la suite $(u_n)_{n\geq n_0}$ est \textcolor{red}{croissante} si : $(\forall n\geq n_0) \ \ u_{n+1} - u_n \geq 0$
        \item On dit que la suite $(u_n)_{n\geq n_0}$ est \textcolor{red}{décroissante} si : $(\forall n\geq n_0) \ \ u_{n+1} - u_n \leq 0$
        \item On dit que la suite $(u_n)_{n\geq n_0}$ est \textcolor{red}{constante} si : $(\forall n\geq n_0) \ \ u_{n+1} = u_n$
    \end{itemize}
\end{Definition}

\begin{exemple}
    \begin{enumerate}
        \item La suite $(n^2)_{n\geq 0}$ est une suite croissante.
        \item La suite $\left(\displaystyle\frac{1}{n}\right)_{n\geq 1}$ est une suite décroissante.
    \end{enumerate}
\end{exemple}
\mysubsection{3}{Suite arithmétique}
\begin{Definition}
    On dit que la suite $\suite$ est \textcolor{red}{arithmétique} s'il existe un réel $r$ (indépendant de $n$) tel que : $$(\forall n\geq n_0) \ \ u_{n+1} - u_n = r$$
    Le nombre $r$ est appelé la raison de la suite $\suite$.
\end{Definition}

&\\
\hline
\end{tabular}

\begin{tabular}{|>{\raggedright\arraybackslash}p{17cm}|>{\centering\arraybackslash}p{0.8cm}|}
\hline
\vspace{1mm}
\begin{Proposition}
    Si la suite $\suite$ est une suite arithmétique de raison $r$, alors pour tout $(n;p)$ tels que $n\geq n_0$ et $p\geq n_0$ : 
    $$u_n = u_p + (n-p).r \ \text{ et } \ u_p + u_{p+1} + \dots + u_n = \displaystyle\frac{n - p + 1}{2}(u_n + u_p) \ \ (n\geq p)$$
\end{Proposition}

\mysubsection{4}{Suite géométrique}
\begin{Definition}
    On dit que la suite $\suite$ est géométrique s'il existe un réel $q$ (indépendant de $n$) tel que : $$(\forall n\geq n_0) \ \ u_{n+1}=qu_{n}$$
    Le nombre $q$ est appelé la raison de la suite $\suite$.
\end{Definition}

\begin{Proposition}
    Si la suite $\suite$ est une suite géométrique de raison $q\in\mathbb{R}^*-\{1\}$, alors pour tout $(n;p)$ tels que $n\geq n_0$ et $p\geq n_0$ :
    $$u_n = u_p.q^{n-p} \text{ \ et \ } u_p + u_{p+1} + \dots + u_n = u_P.\displaystyle\frac{1 - q^{n - p + 1}}{1 - q} \quad (n\geq p)$$
\end{Proposition}

\mysection{Limite d'une fonction numérique}

\mysubsection{1}{Limite infinie des suites usuelles}
\begin{Proposition}
    Soit $p$ un entier naturel supérieure ou égal à $4$. Alors :
    \begin{itemize}
        \item $\Lim_{n\to +\infty} n = +\infty\quad ; \quad \Lim_{n\to +\infty} n^2 = +\infty \quad ; \quad \Lim_{n\to +\infty}n^3 = +\infty \quad ;\quad \Lim_{n\to +\infty}n^p = +\infty$.
        \item $\Lim_{n\to +\infty}\sqrt{n} \quad ; \quad \Lim_{n\to +\infty}\sqrt[3]{n} = +\infty \quad ; \quad \Lim_{n\to +\infty}\sqrt[p]{n} = +\infty$.
    \end{itemize}
\end{Proposition}

\mysubsection{2}{Convergence d'une suite numérique}
\begin{Definition}
    On dit qu'une suite numérique est \textcolor{red}{convergente} si elle admet une limite réelle. Dans le cas contraire, on dit qu'elle est \textcolor{red}{divergente}.
\end{Definition}

\begin{Proposition}
    Soit $k$ un réel et $p$ un entier naturel non nul. Alors : $\Lim_{n\to +\infty}\displaystyle\frac{k}{n^p} = 0$ et $\Lim_{n\to +\infty}\displaystyle\frac{k}{\sqrt{n}} = 0$.

    En particulier : $\Lim_{n\to +\infty}\displaystyle\frac{1}{n} = 0\quad;\quad \Lim_{n\to +\infty}\displaystyle\frac{1}{n^2} = 0\quad;\quad\Lim_{n\to +\infty}\displaystyle\frac{1}{\sqrt{n}} = 0$.
\end{Proposition}
\begin{Proposition}
    Étant donné une suite $\suite$ et $l\in\mathbb{R}$, alors on a les équivalences suivantes :
    $$\Lim_{n\to +\infty}u_n = l \Longleftrightarrow\Lim_{n\to +\infty}(u_n - l) = 0 \quad\text{ et }\quad\Lim_{n\to +\infty}u_n = l \Longleftrightarrow\Lim_{n\to +\infty}|u_n - l| = 0$$
\end{Proposition}
\textcolor{red}{Remarque :}

\begin{itemize}
    \item Dire qu'une suite diverge (ou qu'elle est divergente), ne signifie pas qu'elle tend vers l'infinie.
    
    Cela signifie exactement que la suite n'a pas de limite ou qu'elle tend vers l'infini.
\end{itemize}

&\\
\hline
\end{tabular}

\begin{tabular}{|>{\raggedright\arraybackslash}p{17cm}|>{\centering\arraybackslash}p{0.8cm}|}
\hline

\vspace{1mm}
\begin{exemple}
    Soit $\suite$ une suite numérique définie par : $u_n = \displaystyle\frac{3n^2 + (-1)^n}{n^2}$. Montrons que $\Lim_{n\to +\infty}u_n = 3$.
\end{exemple}

\mysubsection{3}{Opérations sur les limites}
\begin{Proposition}
    \begin{itemize}
        \item Limite de la somme :  
        
        \begin{tabular}{|c|c|c|c|c|c|c|}
        \hline
           $\Lim_{}u_n$  & $l$ & $l$ & $l$ & $+\infty$ & $-\infty$ & $+\infty$  \\ 
        \hline
           $\Lim_{}v_n$  & $l^{'}$ & $+\infty$ & $-\infty$ & $+\infty$ & $-\infty$ & $-\infty$ \\
        \hline
            $\Lim_{}(u_n + v_n)$ & $l + l^{'}$ & $+\infty$ & $-\infty$ & $+\infty$ & $-\infty$ & Forme indéterminée\\
        \hline
        \end{tabular}
        \item Limite du produit :
        
        \begin{tabular}{|c|c|c|c|c|c|c|c|}
        \hline
           $\Lim_{}u_n$  & $l$ & $+\infty$ & $+\infty$ & $-\infty$ & $-\infty$ & $+\infty$ & $+\infty$ ou $-\infty$ \\ 
        \hline
           $\Lim_{}v_n$  & $l^{'}$ & $l^{'}>0$ & $l^{'} < 0$ & $+\infty$ & $- \infty$ & $+\infty$ & $0$\\
        \hline
            $\Lim_{}(u_n \times v_n)$ & $ll^{'}$ & $+\infty$ & $-\infty$ & $-\infty$ & $+\infty$ & $+\infty$ & Forme indéterminée\\
        \hline
        \end{tabular}
        \item Limite de l'inverse :
        
        \begin{tabular}{|c|c|c|c|}
        \hline
            $\Lim_{}u_n$ & $l\neq 0$ & $+\infty$ ou $-\infty$ & $0$ \\
        \hline
            $\Lim_{}\dfrac{1}{u_n}$ & $\displaystyle\frac{1}{l}$ & $0$ & $\Lim{}\dfrac{1}{|u_n|} = +\infty$ \\
        \hline
        \end{tabular}
    \end{itemize}
\end{Proposition}

\begin{application}
    Calculer la limite de chacune des suites suivantes définies par :
    $$u_n = \displaystyle\frac{2\sqrt{n} - 3}{5\sqrt{n} + 2}\quad;\quad v_n = \dfrac{n^2 + 5n + 6}{n^2 + 7n + 8}\quad;\quad w_n = \dfrac{2n-7}{n^4 + 3n^2 + 1}\quad;\quad x_n = \sqrt{4n^2 - 4n + 1} - 2n$$
    $$y_n = \sqrt{n+2} - \sqrt{n+3} \quad;\quad a_n = (n+4)\sqrt[3]{\dfrac{1}{n^3}} \quad;\quad b_n = 4 + \dfrac{1}{n} - \dfrac{10}{\sqrt[5]{n}} $$
\end{application}

\mysubsection{4}{Limites et ordre}
\begin{Proposition}
    Soit $\suite$ et $(v_n)_{n\geq n_0}$ deux suites numériques convergentes. Alors :
    \begin{itemize}
        \item Si la suite $\suite$ est positive, alors : $\Lim u_n \geq 0$.
        \item Si $u_n \leq v_n$ pour tout entier $n\geq n_0$, alors : $\Lim u_n \leq \Lim v_n$.
    \end{itemize}
\end{Proposition}

\textcolor{red}{Remarque :}

Une suite strictement positive (à partir d'un certain rang) et convergente peut avoir une limite nulle. Par exemple : On a : $\dfrac{1}{n}$ pour tout $n\in\mathbb{N}^*$, mais $\Lim \dfrac{1}{n} = 0$.

\mysubsection{5}{Monotonie et convergence}

\begin{Thm}{Théorème}
    \begin{itemize}
        \item Toute suite croissante majorée est convergente.
        \item Toute suite décroissante minorée est convergente.
    \end{itemize}
    Ce résultat porte le nom de \textcolor{red}{Théorème de la convergence monotone}.
\end{Thm}
&\\
\hline
\end{tabular}

\begin{tabular}{|>{\raggedright\arraybackslash}p{17cm}|>{\centering\arraybackslash}p{0.8cm}|}
\hline
\vspace{1mm}

\begin{application}
    \begin{enumerate}
        \item Soit $\suite$ une suite numérique défine par $u_0 = 0$ et pour tout $n\in\mathbb{N}$ : $u_{n+1}=\sqrt{2+u_n}$.
            \begin{enumerate}
                \item Montrer par récurrence que pour tout $n\in\mathbb{N}$ : $0\leq u_n\leq 2$.
                \item Étudier la monotonie de la suite $\suite$.
                \item Déduire que la suite $\suite$ est convergente.
            \end{enumerate}
        \item Soit $\suite$ une suite numérique définie par $u_0 = \dfrac{3}{2}$ et pour tout $n\in\mathbb{N}$ : $u_{n+1} = \dfrac{u_n^2 + 1}{2u_n}$.
        \begin{enumerate}
            \item Montrer par récurrence que $(\forall n\in\mathbb{N}) \ \ u_n > 1$.
            \item Érudier la monotonie de la suite $\suite$.
            \item Déduire que la suite $\suite$  est convergente.
        \end{enumerate}
    \end{enumerate}
\end{application}

\mysection{Critére de convergence - Limite d'une suite géométrique}

\mysubsection{1}{Critéres de convergence}
\begin{Proposition}
    Soit $\suite$ et $(v_n)_{n\geq n_0}$ deux suites numériques telles que pour tout $n\geq n_0$ : $u_n \leq v_n$.
    \begin{itemize}
        \item Si $\Lim_{n\to +\infty}u_n = +\infty$ alors $\Lim_{n\to +\infty}v_n = +\infty$
        \item Si $\Lim_{n\to +\infty}v_n = -\infty$ alors $\Lim_{n\to +\infty}u_n = -\infty$.
    \end{itemize}
\end{Proposition}
\begin{exemple}
    \begin{enumerate}
        \item Soit $(u_n)$ la suite définie par : $u_n = -n + \cos\sqrt{n} - 1$.
        Montrons que $\Lim u_n = -\infty$.
    \end{enumerate}
\end{exemple}

\begin{application}
    Soit $(u_n)_{n\geq 0}$ la suite numérique définie par $u_0 = 1$ et pour tout $n\in\mathbb{N}$ : $u_{n+1} = u_n + 2n +3$.
    \begin{enumerate}
        \item Montrer que $(\forall n\in\mathbb{N}) \ \ u_n > n^2$.
        \item Déduire la limite de $u_n$.
    \end{enumerate}
\end{application}

\begin{Thm}{Théorème}
    Soit $\suite$ et $(v_n)_{n\geq n_0}$ deux suites numériques convergeant vers une limite commune $l$.

    Si $(\exists N\geq n_0)\ (\forall n\geq N) \ \ v_n \leq u_n \leq w_n$, alors $\Lim_{n\to +\infty}u_n = l$.

    Ce résultat est appelé \textcolor{red}{Théorème des limites comparées}.
\end{Thm}

\begin{exemple}
    Soit $(u_n)_{n\geq 0}$ la suite définie par : $u_n = \dfrac{n + \cos{n}}{n^2 + 1}$.
    
    Montrons que la suite $(u_n)$ est convergente et calculons sa limite.
\end{exemple}

    
&\\
\hline
\end{tabular}

\begin{tabular}{|>{\raggedright\arraybackslash}p{17cm}|>{\centering\arraybackslash}p{0.8cm}|}
\hline
\vspace{1mm}

\begin{application}
\begin{enumerate}
    \item Soit $(u_n)_{n\geq 1}$ la suite numérique définie par : $u_n = \dfrac{3n + \cos{\dfrac{5}{n}}}{n + 1}$.
        \begin{enumerate}
            \item Montrer que : $(\forall n\in\mathbb{N}^*) \ \ \dfrac{3n-1}{n+1}\leq u_n\leq \dfrac{3n + 1}{n + 1}$
            \item En déduire que la suite $(u_n)_{n\geq 1}$ est convergente et déterminer sa limite.
        \end{enumerate}
    \item Soit $(v_n)_{n\geq 0}$ la suite numérique définie par : $v_n = \dfrac{\sqrt{n}\sin{n}}{n + 1}$.
        \begin{enumerate}
            \item Montrer que la suite $(v_n)_{n\geq 0}$ est convergente st calculer sa limite.
        \end{enumerate}
\end{enumerate}
\end{application}

\textcolor{red}{Corollaire :}

Soit $\suite$ une suite numérique et $l$ un nombre réel. S'il existe une suite $(v_n)_{n\geq n_0}$ tendant vers $0$ telle que pour tout $n\geq n_0$, $|u_n - l|\leq v_n$, alors $\Lim_{n\to +\infty}u_n = l$.

\mysubsection{2}{Limite d'une suite géométrique}
\begin{Proposition}
    Soit $q$ un nombre réel non nul.
    \begin{itemize}
        \item Si $q>1$ alors $\Lim_{n\to +\infty}q^n = +\infty$.
        \item Si $-1 < q < 1$ alors $\Lim_{n\to +\infty}q^n = 0$.
        \item Si $q = 1$ alors $\Lim_{n\to +\infty}q^n = 1$.
        \item Si $q \leq -1$ alors la suite $(q^n)_{n\geq 0}$ n'admet pas de limite.
    \end{itemize}
\end{Proposition}

\begin{exemple}
    $\Lim_{n\to +\infty}\left(\dfrac{1}{2}\right)^n = 0$ car : $-1 < \dfrac{1}{2} < 1$.

    $\Lim_{n\to +\infty}\left(- \dfrac{1}{4}\right)^n = 0$ car : $-1 < -\dfrac{1}{4} < 1$. 
\end{exemple}

\begin{application}
    Calculer la limite de $(u_n)$ :
    $$u_n = \left(- \dfrac{4}{5}\right)^n\quad;\quad u_n = \dfrac{2^n + 5^n}{ 3^n - 5^n}\quad;\quad u_n = \dfrac{4^n}{3^{2n}}\quad;\quad u_n = \dfrac{3^{2n+1}}{2^{n+1}\times 5^{n - 2}} \quad;\quad u_n = \left(\dfrac{1 -\sqrt{2}}{1+\sqrt{2}}\right)^n$$
\end{application}

\mysection{Suites de la forme $u_{n+1}=f(u_n)$ et $v_n = f(u_n)$}

\mysubsection{1}{Limite d'une suite de la forme $v_n=f(u_n)$}
\begin{Proposition}
    Si une suite $\suite$ est convergente vers $l$ et $f$ est une fonction continue en $l$, alors la suite $(v_n)_{n\geq n_0}$ définie par $v_n = f(u_n)$ est convergente et sa limite est $f(l)$.
\end{Proposition}

\begin{exemple}
    Soit $(v_n)$ la suite numérique définie par : $v_n = \tan\left(\dfrac{\pi n + 1}{3n + 2}\right)$. Montrons que $\Lim v_n = \sqrt{3}$.
\end{exemple}
&\\
\hline
\end{tabular}


\begin{tabular}{|>{\raggedright\arraybackslash}p{17cm}|>{\centering\arraybackslash}p{0.8cm}|}
\hline
\vspace{1mm}
\begin{application}
    Déterminer les limites des suites numériques définies par :
    $$u_n = \cos\left(\dfrac{n\pi - 2}{n^2 + 1}\right)\quad;\quad v_n = \sqrt[3]{\dfrac{8n^2 + 1}{27n^2 - 2n + 5}}\quad;\quad w_n = \sin\left(\dfrac{\pi^2}{n + \pi}\right)$$
\end{application}

\mysubsection{2}{Limite de la suite $(n^r)$ où $r\in\mathbb{Q}^*$}
\begin{Proposition}
    Soit $r$ un nombre rationnel non nul.
    \begin{itemize}
        \item Si $r>0$ alors $\Lim_{n\to +\infty}n^r = +\infty$.
        \item Si $r<0$ alors $\Lim_{n\to +\infty}n^r = 0$.
    \end{itemize}
\end{Proposition}
\begin{exemple}
    $$\Lim_{n\to +\infty}n^{\frac{5}{6}} = +\infty \quad;\quad \Lim_{n\to +\infty}n^{\frac{2}{7}} = +\infty \quad;\quad \Lim_{n\to +\infty}n^{-\frac{5}{6}} = 0\quad;\quad \Lim_{n\to +\infty}n^{-\frac{54}{61}} = 0$$
\end{exemple}
\mysubsection{3}{Suite de la forme $u_{n+1} = f(u_n)$}
\begin{Proposition}
    Soit $f$ une fonction continue sur un intervalle $I$ telle que $f(I)\subset I$.

    Soit $\suite$ une suite réelle définie par $u_{n_0}\in I$ et $u_{n+1} = f(u_n)$ pour tout entier $n\geq n_0$.

    Si $\suite$ est convergente de limite $l$ et $l\in I$, alors $l$ est solution dans $I$ de l'équation $f(x) = x$.
\end{Proposition}

\begin{application}
    Soit $(u_n)$ la suite définie par : $u_0 = 1$ et $u_{n+1} = \sqrt{\dfrac{1+ u_n}{2}}$ pour tout $n\in\mathbb{N}$.
    \begin{enumerate}
        \item Montrer par récurrence que : $0\leq u_n \leq 1$ pour tout $n\in\mathbb{N}$.
        \item Montrer que $(u_n)_{n\geq 0}$ est décroissante.
        \item En déduire que la suite $(u_n)_{n\geq 0}$ est convergente et déterminer sa limite.
    \end{enumerate}
\end{application}

&\\
\hline
\end{tabular}


\end{document}
