\documentclass[12pt,a4paper]{article}
\usepackage[top=2cm,left=1.5cm,right=1cm,bottom=2cm]{geometry}
\usepackage{amssymb,mathtools,amsthm}
\usepackage{fourier}
\usepackage{xcolor}
\usepackage{multicol, array, fancyhdr}
\usepackage{tasks}
\newcommand{\Lim}{\displaystyle\lim}
\renewcommand{\columnseprule}{1pt}
\renewcommand{\arraystretch}{1.5}
% \renewcommand{\frac}[2]{\displaystyle\frac{#1}{#2}}


%======================================================
\newtheoremstyle{mystyle}
{\topsep}% espace avant
{\topsep}% espace après
{\upshape}% police du corps du théorème
{}% indentation (vide pour rien, \parindent)
{\bfseries\sffamily}% police du titre du théorème
{ :}% ponctuation après le théorème
{ }% après le titre du théorème (espace ou \newline)
{%
    \rule[0.5\baselineskip]{0.5\textwidth}{1pt}%
    \newline\fcolorbox{black}{white}{%
    \thmname{#1}\thmnumber{ \textup{#2}}\thmnote{ \textnormal{(#3)}}%
}%
\medskip%
}% spécifications du titre

\theoremstyle{mystyle}
\newtheorem{exo}{Exercice}


%======================================================



\begin{document}


\pagestyle{fancy}
\fancyhf{} % clear all header and footer fields
\fancyhead[L]{Lycée : Zitoun \hspace{1.5cm} Année scolaire : 2024-2025} % Left header
\fancyhead[C]{ \hspace{4cm} Niveau : 2BAC PC} % Right-Center header
\fancyhead[R]{Prof. Othmane Laksoumi} % Right header
\fancyfoot[C]{\thepage} % Footer


\begin{center}
    \textbf{\Large Dérivation et étude de fonctions numériques}
\end{center}
\begin{multicols*}{2}

\begin{exo}
Calculer les limites suivantes :
\begin{enumerate}
	\item $\Lim_{n \to +\infty} n^3 + \frac{3}{\sqrt{n}} - \dfrac{1}{n^8} + 8 \ \ ; \ \ \Lim_{n \to +\infty} -n^3 + 6n^2 - n + 9$
	\item $\Lim_{n \to +\infty} \frac{-2n + n^2 + 1}{n^3 - 6n^2 + n - 1} \ \ ; \ \ \Lim_{n \to +\infty} n - 5\sqrt{n}$
	\item $\Lim_{n \to +\infty} n^2 - n + (-1)^n \ \ ; \ \ \Lim_{n \to +\infty} \frac{5 +2\sqrt{n}}{2 - 3\sqrt{n}}$
	\item $\Lim_{n \to +\infty} \sqrt{n^2 + 1} - n \ \ ; \ \ \Lim_{n \to +\infty} (n - 2) \frac{1}{\sqrt[5]{n^3}}$
\end{enumerate}
\end{exo}

\begin{exo}
Soit \( (u_n) \) la suite numérique définie par :
\[
u_0 = -1 \quad \text{et} \quad u_{n+1} = \frac{9}{6 - u_n}
\]
\begin{enumerate}
	\item Montrer par récurrence que : \( (\forall n \in \mathbb{N}) : u_n < 3 \).
	\item Étudier la monotonie de la suite \( (u_n) \).
	\item On considère la suite \( (v_n) \) définie par :
$$(\forall n \in \mathbb{N}) : v_n = \frac{1}{u_n - 3}$$
		\begin{enumerate}
			\item Montrer que \( (v_n) \) est une suite arithmétique. Préciser la raison et le premier terme.
			\item Exprimer \( v_n \) puis \( u_n \) en fonction de \( n \).
		\end{enumerate}
\end{enumerate}
	\item
		\begin{enumerate}
			\item  Calculer en fonction de \( n \) la somme : $$S_n = v_1 + v_2 + \ldots + v_n$$
			\item En déduire la limite : \( \Lim_{n \to +\infty} S_n \).
		\end{enumerate}
\end{exo}

\begin{exo}

On considère la suite $(u_n)$ définie par $ u_0 = 1 $ et pour tout $ n \in \mathbb{N} $ :  
$ u_{n+1} = \frac{3u_n + 2}{u_n + 2} $  

1) Montrer par récurrence que : $ (\forall n \in \mathbb{N}) \, 1 \leq u_n < 2 $.  

2) Montrer que :  
$ (\forall n \in \mathbb{N}) \, u_{n+1} - u_n = \frac{-(u_n + 1)(u_n - 2)}{u_n + 2}. $  

3) Étudier la monotonie de la suite $(u_n)$.  

4) En déduire que la suite $(u_n)$ est convergente.  

\end{exo}




















\end{multicols*}



\end{document}