\documentclass[10pt,a4paper]{article}
\usepackage[right=0.5cm, left=0.5cm,top=0.5cm,bottom=0.5cm]{geometry}
\usepackage{enumitem}
\usepackage{forest}
\usepackage{amsfonts}
\usepackage{tikz}
\usetikzlibrary{positioning}
\usepackage{graphicx}
\usepackage{array, tasks}
\usepackage{blindtext}
\usepackage{fontspec}
\usepackage{amsmath,amsfonts,amssymb,mathrsfs,amsthm}
\usepackage{fancyhdr}
\usepackage{xcolor}
\usepackage{booktabs}
\usepackage[font={bf}]{caption}
% \captionsetup[table]{box=colorbox,boxcolor=orange!20}
\usepackage{float}
\usepackage{esvect}
\usepackage{tabularx}
\usepackage{pifont}
\usepackage{colortbl}
 \usepackage{fancybox}
 \mathversion{bold}
 \usepackage{pgfplots}
 % \usepackage[utf8]{inputenc}
\usepackage{tikz}
 \usepackage[tikz]{bclogo}%
 \usepackage{mathpazo}
\usepackage{ulem}
\usepackage{yagusylo}
\usepackage{textcomp}\usepackage{blindtext}
\usepackage{multicol}
\usepackage{varwidth}
\usetikzlibrary{calc,intersections}
\usepackage{pgfplots}
%\usepackage{fourier}
\pgfplotsset{compat=1.11}
\usepackage{tkz-tab}
\usepackage{xcolor}
\usepackage{color}
\usetikzlibrary{calc}
\mathchardef\times="2202
\usepackage[most]{tcolorbox}
\definecolor{lightgray}{gray}{0.9}
\definecolor{ocre}{RGB}{0,244,244} 
\definecolor{head}{RGB}{255,211,204}
\definecolor{browndark}{RGB}{105,79,56}
%\RequirePackage[framemethod=default]{mdframed}
\usepackage{tikz}
\usetikzlibrary{calc,patterns,decorations.pathmorphing,arrows.meta,decorations.markings}
\usetikzlibrary{arrows.meta}
\makeatletter
\tcbuselibrary{skins,breakable,xparse}
\tcbset{%
  save height/.code={%
    \tcbset{breakable}%
    \providecommand{#1}{2cm}%
    \def\tcb@split@start{%
      \tcb@breakat@init%
      \tcb@comp@h@page%
      \def\tcb@ch{%
        \tcbset{height=\tcb@h@page}%
        \tcbdimto#1{#1+\tcb@h@page-\tcb@natheight}%
        \immediate\write\@auxout{\string\gdef\string#1{#1}}%
        \tcb@ch%
      }%
      \tcb@drawcolorbox@standalone%
    }%
  }%
}
\newcommand{\Lim}{\displaystyle\lim}
\makeatother
\newcommand{\oij}{$\left( \text{O};\vv{i},\vv{j} , \vv{k}\right)$}
\colorlet{darkred}{red!30!black}
\newcommand{\red}[1]{\textcolor{darkred}{ #1}}
\newcommand{\rr}{\mathbb{R}}
\renewcommand{\baselinestretch}{1.2}
 \setlength{\arrayrulewidth}{1.25pt}
\usepackage{titlesec}
\usepackage{titletoc}
\usepackage{minitoc}
\usepackage{ulem}
%--------------------------------------------------------------

\usetikzlibrary{decorations.pathmorphing}
\tcbuselibrary{skins}

%%%%%%%%%%%
%-------------------------------------------------------------------------
\tcbset{
        enhanced,
        colback=white,
        boxrule=0.1pt,
        colframe=brown!10,
        fonttitle=\bfseries
       }
\definecolor{problemblue}{RGB}{100,134,158}
\definecolor{idiomsgreen}{RGB}{0,162,0}
\definecolor{exercisebgblue}{RGB}{192,232,252}
\definecolor{darkbrown}{rgb}{0.4, 0.26, 0.13}

\newcommand*{\arraycolor}[1]{\protect\leavevmode\color{#1}}
\newcolumntype{A}{>{\columncolor{blue!50!white}}c}
\newcolumntype{B}{>{\columncolor{LightGoldenrod}}c}
\newcolumntype{C}{>{\columncolor{FireBrick!50}}c}
\newcolumntype{D}{>{\columncolor{Gray!42}}c}

\newcounter{mysection}
\newcounter{mysubsection}
\newcommand{\mysection}[1]{%
    \stepcounter{mysection} % Increment the counter
    \textcolor{red}{\LARGE\themysection. #1 :}
}
\newcommand{\mysubsection}[2]{
    \stepcounter{mysubsection}
    \textcolor{red}{\large \themysection.#1. #2 :}
}
% \textcolor{red}{\LARGE\bfseries 1. Les équation du deuxiéme degrée :}

%------------------------------------------------------
\newtcolorbox[auto counter]{Definition}{enhanced,
before skip=2mm,after skip=2mm,
colback=yellow!20!white,colframe=lime,boxrule=0.2mm,
attach boxed title to top left =
    {xshift=0.6cm,yshift*=1mm-\tcboxedtitleheight},
    varwidth boxed title*=-3cm,
    boxed title style={frame code={
                        \path[fill=lime]
                            ([yshift=-1mm,xshift=-1mm]frame.north west)  
                            arc[start angle=0,end angle=180,radius=1mm]
                            ([yshift=-1mm,xshift=1mm]frame.north east)
                            arc[start angle=180,end angle=0,radius=1mm];
                        \path[left color=lime,right color = lime,
                            middle color = lime]
                            ([xshift=-2mm]frame.north west) -- ([xshift=2mm]frame.north east)
                            [rounded corners=1mm]-- ([xshift=1mm,yshift=-1mm]frame.north east) 
                            -- (frame.south east) -- (frame.south west)
                            -- ([xshift=-1mm,yshift=-1mm]frame.north west)
                            [sharp corners]-- cycle;
                            },interior engine=empty,
                    },
fonttitle=\bfseries\sffamily,
title={Definition ~\thetcbcounter}}
%------------------------------------------------------
\newtcolorbox[auto counter]{Proposition}{enhanced,
before skip=2mm,after skip=2mm,
colback=yellow!20!white,colframe=blue,boxrule=0.2mm,
attach boxed title to top left =
    {xshift=0.6cm,yshift*=1mm-\tcboxedtitleheight},
    varwidth boxed title*=-3cm,
    boxed title style={frame code={
                        \path[fill=blue]
                            ([yshift=-1mm,xshift=-1mm]frame.north west)  
                            arc[start angle=0,end angle=180,radius=1mm]
                            ([yshift=-1mm,xshift=1mm]frame.north east)
                            arc[start angle=180,end angle=0,radius=1mm];
                        \path[left color=blue,right color = blue,
                            middle color = blue]
                            ([xshift=-2mm]frame.north west) -- ([xshift=2mm]frame.north east)
                            [rounded corners=1mm]-- ([xshift=1mm,yshift=-1mm]frame.north east) 
                            -- (frame.south east) -- (frame.south west)
                            -- ([xshift=-1mm,yshift=-1mm]frame.north west)
                            [sharp corners]-- cycle;
                            },interior engine=empty,
                    },
fonttitle=\bfseries\sffamily,
title={Proposition ~\thetcbcounter}}
%------------------------------------------------------
\newtcolorbox[auto counter]{Thm}[1]{enhanced,
before skip=2mm,after skip=2mm,
colback=yellow!20!white,colframe=red,boxrule=0.2mm,
attach boxed title to top left =
    {xshift=0.6cm,yshift*=1mm-\tcboxedtitleheight},
    varwidth boxed title*=-3cm,
    boxed title style={frame code={
                        \path[fill=red]
                            ([yshift=-1mm,xshift=-1mm]frame.north west)  
                            arc[start angle=0,end angle=180,radius=1mm]
                            ([yshift=-1mm,xshift=1mm]frame.north east)
                            arc[start angle=180,end angle=0,radius=1mm];
                        \path[left color=red,right color = red,
                            middle color = red]
                            ([xshift=-2mm]frame.north west) -- ([xshift=2mm]frame.north east)
                            [rounded corners=1mm]-- ([xshift=1mm,yshift=-1mm]frame.north east) 
                            -- (frame.south east) -- (frame.south west)
                            -- ([xshift=-1mm,yshift=-1mm]frame.north west)
                            [sharp corners]-- cycle;
                            },interior engine=empty,
                    },
fonttitle=\bfseries\sffamily,
title={#1 ~\thetcbcounter}}
%------------------------------------------------------
\newtcolorbox[auto counter]{Exemple}{
  % breakable,1
  enhanced,
  colback=white,
  boxrule=0pt,
  arc=0pt,
  outer arc=0pt,
  title=Exemple ~\thetcbcounter,
  fonttitle=\bfseries\sffamily\large\strut,
  coltitle=problemblue,
  colbacktitle=problemblue,
  title style={
left color=exercisebgblue,
    right color=white,
    middle color=exercisebgblue  
  },
  overlay={
    \draw[line width=1pt,problemblue] (frame.south west) -- (frame.south east);
    \draw[line width=1pt,problemblue] (frame.north west) -- (frame.north east);
    \draw[line width=1pt,problemblue] (frame.south west) -- (frame.north west);
    \draw[line width=1pt,problemblue] (frame.south east) -- (frame.north east);
  }
}
%----------------------------------------------------
\newtcolorbox[auto counter]{application}{
  % breakable,
  enhanced,
  colback=white,
  boxrule=0pt,
  arc=0pt,
  outer arc=0pt,
  title=Application ~\thetcbcounter,
  fonttitle=\bfseries\sffamily\large\strut,
  coltitle=problemblue,
  colbacktitle=problemblue,
  title style={
left color=exercisebgblue,
    right color=white,
    middle color=exercisebgblue  
  },
  overlay={
    \draw[line width=1pt,problemblue] (frame.south west) -- (frame.south east);
    \draw[line width=1pt,problemblue] (frame.north west) -- (frame.north east);
    \draw[line width=1pt,problemblue] (frame.south west) -- (frame.north west);
    \draw[line width=1pt,problemblue] (frame.south east) -- (frame.north east);
  }
}
%----------------------------------------------------
\newtcolorbox[auto counter]{Activite}{
  % breakable,
  enhanced,
  colback=white,
  boxrule=0pt,
  arc=0pt,
  outer arc=0pt,
  title=Activité ~\thetcbcounter,
  fonttitle=\bfseries\sffamily\large\strut,
  coltitle=problemblue,
  colbacktitle=problemblue,
  title style={
left color=yellow!50!white,
    right color=white,
    middle color=yellow!20!white  
  },
  overlay={
    \draw[line width=1pt,problemblue] (frame.south west) -- (frame.south east);
    \draw[line width=1pt,problemblue] (frame.north west) -- (frame.north east);
    \draw[line width=1pt,problemblue] (frame.south west) -- (frame.north west);
    \draw[line width=1pt,problemblue] (frame.south east) -- (frame.north east);
  }
}
%---------------------------------------------
\newtcolorbox{mybox}[2]{enhanced,breakable,
    before skip=2mm,after skip=2mm,
    colback=white,colframe=#2!30!blue,boxrule=0.3mm,rightrule=0.3mm,
    attach boxed title to top center={xshift=0cm,yshift*=1mm-\tcboxedtitleheight},
    varwidth boxed title*=-3cm,
    boxed title style={frame code={
    \path[fill=#2!30!black]
    ([yshift=-1mm,xshift=-1mm]frame.north west)
    arc[start angle=0,end angle=180,radius=1mm]
    ([yshift=-1mm,xshift=1mm]frame.north east)
    arc[start angle=180,end angle=0,radius=1mm];
    \path[draw=black,line width=1pt,left color=#2!1!white,right color=#2!1!blue!65,
    middle color=#2!1!green]
    ([xshift=-2mm]frame.north west) -- ([xshift=2mm]frame.north east)
    [rounded corners=1mm]-- ([xshift=1mm,yshift=-1mm]frame.north east)
    -- (frame.south east) -- (frame.south west)
    -- ([xshift=-1mm,yshift=-1mm]frame.north west)
    [sharp corners]-- cycle;
    },interior engine=empty,
    },
title=#1,coltitle=black,fonttitle=\sffamily}
%---------------------------------------------
\newtcolorbox{boxone}{%
    enhanced,
    colback=brown!10,
    boxrule=0pt,
    sharp corners,
    drop lifted shadow,
    frame hidden,
    fontupper=\bfseries,
    notitle,
    overlay={%
        \draw[Circle-Circle, brown!70!black, line width=2pt](frame.north west)--(frame.south west); 
        \draw[Circle-Circle, brown!70!black, line width=2pt](frame.north east)--(frame.south east);}
    }

    \tikzset{ midd/.style={
decoration={
markings,
mark=at position 0.5 with {%
\makebox[4pt][r]{\arrow{triangle 45}},
},
},
postaction={decorate},{Circle[]}-{Circle[]}
},}%
\tikzset{ midd/.style={
decoration={
markings,
% mark=at position 0.5 with {%
% \makebox[4pt][r]{\arrow{triangle 45}},
% },
},
% postaction={decorate},{Circle[]}-{Circle[]}
},}%
    
\begin{document}

\begin{tcolorbox}[title=\textcolor{blue}{\shadowbox{ Prof : Othmane Laksoumi}}
\hfill
\textcolor{blue}{\shadowbox{Les fonctions primitives}}]

\end{tcolorbox}

\begin{mybox}{Lycée Qualifiant Zitoun}{gray}
    \begin{minipage}{8cm}
    \textcolor{darkbrown}{Année scolaire : } 2024-2025 \\
    \textcolor{darkbrown}{Niveau : } 2 Bac Sciences Physiques \\
    \textcolor{darkbrown}{Durée totale : } $2h$
    \end{minipage}
\end{mybox}

\begin{boxone}
{\Large\ding{45}}
\textcolor{red}{\large Contenus du programme :}
\begin{itemize}
   \item Fonctions primitives d'une fonction continue sur un intervalle;
   \item Fonctions primitives de la somme de deux fonctions; fonctions primitives du produit d'une fonction par un nombre réel.
\end{itemize}

{\Large\ding{45}}
\textcolor{red}{\large Les capacités attendues :}
\begin{itemize}
    \item Déterminer les fonctions primitives des fonction usuelles;
    \item Utiliser les formules de dérivation pour déterminer les fonctions primitives d'une fonction sur un intervalle.
\end{itemize}

{\Large\ding{45}}
\textcolor{red}{\large Recommandations pédagogiques :} 
  \begin{itemize}
    \item On déterminera les fonctions primitives des fonctions usuelles à partir de la lecture croisée du tableau des dérivées de ces fonctions.
  \end{itemize}
\end{boxone}


\begin{tabular}{|>{\raggedright\arraybackslash}p{17cm}|>{\centering\arraybackslash}p{0.8cm}|}
\hline
\rowcolor{head}

\centering Contenu du cour &
 Durée \\
\hline
 
\vspace{1mm}
\mysection{Primitive d'une fonction sur un intervalle}

\begin{Definition}
Soit $f$ et $F$ deux fonctions définies sur un intervalle $I$ de $\mathbb{R}$.

On dit que la fonction $F$ est une primitive de la fonction $f$ sur $I$ si :
$$F \text{est dérivable sur } I \text{et pour tout } x\in I \ : \ F^{\prime}(x) = f(x)$$
\end{Definition}


\begin{Exemple}
On considére les fonctions $f$ et $F$ définies sur l'intervalle $I = ]0;+\infty[$ par :
$$f(x) = 1 + 2x + \dfrac{1}{\sqrt{x}} \text{ et } F(x) = x + x^2 + 2\sqrt{x}$$
La fonction $F$ est une primitive de la fonction $f$ sur $I$ car $F$ est dérivable sur $I$ et pour tout $x\in I$ :
$$F^{\prime}(x) = (x + x^2 + 2\sqrt{x})^{\prime} = 1 + 2x + \dfrac{1}{\sqrt{x}} = f(x)$$
\end{Exemple}

\mysection{Primitive d'une fonction continue}

\begin{Proposition}
Toute fonction continue sur un intervalle $I$ admet une primitive définie sur cet intervalle.
\end{Proposition}

\begin{Proposition}
Soit $f$ une fonction continue sur un intervalle $I$ de $\mathbb{R}$.
\begin{itemize}
	\item Si $F$ est une primitive de la fonction $f$ sur $I$, alors les primitives de $f$ sont les fonctions $x\mapsto F(x) + c$ où $c$ est une constante réelle.
	\item Pour tout $x_0\in I$ et $y_0\in\mathbb{R}$, il existe une unique primitive $G$ de $f$ sur $I$ vérifiant : $G(x_0) = y_0$.
\end{itemize}
\end{Proposition}

\begin{Exemple}
Soit $f$ la fonction définie sur l'intervalle $I = ]0;+\infty[$ par : $f(x) = 1 + 2x + \dfrac{1}{\sqrt{x}}$. On a la fonction $F$ défnine sur $I$ par : $F(x) = x + x^2 + 2\sqrt{x}$ est une fonction dérivable de $f$ sur $I$.\\
Donc les primitives de la fonction $f$ sur l'intervalle $I$ sont les fonction $x\mapsto x + x^2 + 2\sqrt{x} + c$ où $c$ une constante réel.
Soit $G$ la primitive de la fonction $f$ sur $I$ qui s'annule en $1$.

Donc $G(x) = x + x^2  + 2\sqrt{x} + c$ avec $G(1) = 0$.

On a alors $G(1) = 0 \Longleftrightarrow 1 + 1^2 + 2\sqrt{1} + c = 0\Longleftrightarrow 4 + c = 0\Longleftrightarrow c = -4$.\\
Ainsi : $$(\forall x\in]0;+\infty[) ; G(x) = x + x^2 + 2\sqrt{x} - 4$$
\end{Exemple}

\mysection{Opérations sur les primitives}
\begin{Proposition}
Si $F$ et $G$ sont respectivement des primitives des fonction $f$ et $g$ sur un intervalle  $I$ alors :
\begin{itemize}
	\item $F+G$ est une primitive de la fonction $f+g$ sur l'intervalle $I$.
	\item Pour tout $(\alpha;\beta)\in\mathbb{R}^2$, $\alpha F + \beta G$ est une primitive de $\alpha f + \beta g$ sur l'intervalle $I$.
\end{itemize}
\end{Proposition}
&\\
\hline
\end{tabular}


\begin{tabular}{|>{\raggedright\arraybackslash}p{17cm}|>{\centering\arraybackslash}p{0.8cm}|}
\hline
\vspace{1mm}
\mysection{Primitives usuelles}

\begin{tabular}{|>{\centering}m{0.25\textwidth}|>{\centering}m{0.25\textwidth}|>{}m{0.25\textwidth}|}
\hline
    La fonction $f$ & Les primitives $F$ de $f^{\prime}$ & L'intervalle $I$\\
\hline
     $0$ & $c \ \ (c\in\mathbb{R})$ & $\mathbb{R}$ \\
\hline
     $x\mapsto a \ (a\in\mathbb{R}^*)$ & $x\mapsto ax + c$ & $\mathbb{R}$ \\
\hline
     $x\mapsto x^n \ (n\in\mathbb{N}^*)$ & $x\mapsto \dfrac{x^{n+1}}{n+1} + c$ & $\mathbb{R}$ \\
\hline
     $x\mapsto \dfrac{1}{x^n} \ (n\in\mathbb{N}^* - \{1\})$ & $x\mapsto \dfrac{1}{(1-n)x^{n - 1}} + c$ & $\mathbb{R}^*_{+}$ ou $\mathbb{R}^*_{-}$ \\
\hline
     $x\mapsto x^r \ (n\in\mathbb{Q}^* - \{1\})$ & $x\mapsto \dfrac{x^{r+1}}{r+1} + c$ & $\mathbb{R}^*_+$ \\
\hline
	$x\mapsto \sqrt[n]{x} \ (n\in\mathbb{N}^{*})$ & $x\mapsto \dfrac{n}{n+1}\sqrt[n]{x^{n+1}} + c$ & $\mathbb{R}^{+}$ \\
\hline
     $x\mapsto \sin{x}$ & $x\mapsto -\cos{x} + c$ & $\mathbb{R}$ \\
\hline
     $x\mapsto \cos{x}$ & $x\mapsto\sin{x} + c$ & $\mathbb{R}$ \\
\hline
     $x\mapsto 1 + \tan^{2}{x} = \dfrac{1}{\cos^{2}(x)}$ & $x\mapsto \tan{x}$ & $\left]-\displaystyle\frac{\pi}{2} + k\pi;\displaystyle\frac{\pi}{2} + k\pi\right[ \ (k\in\mathbb{Z})$ \\
\hline
     $x\mapsto \sin(ax + b) \ (a\in\mathbb{R}^*)$ & $x\mapsto -\dfrac{1}{a}\cos(ax + b)$ & $\mathbb{R}$ \\
\hline
     $x\mapsto \cos(ax + b) \ (a\in\mathbb{R}^*)$ & $x\mapsto \dfrac{1}{a}\sin(ax + b)$ & $\mathbb{R}$ \\
\hline
	$u^{\prime}v + uv^{\prime}$ & $uv + c$ & Intervalle où  $u$ et $v$ sont dérivables \\
\hline
	$\dfrac{-u^{\prime}}{u^2}$ & $\dfrac{1}{u} + c$ & Intervalle où $u$ est dérivable et ne s'annule pas\\
\hline
	$\dfrac{u}{2\sqrt{u}}$ & $\sqrt{u} + c$ & Intervalle où $u$ est dérivable et strictement positive\\
\hline
	$u^{\prime}u^r \ (r\in\mathbb{Q} ^*- \{1\})$ & $\dfrac{u^{r + 1}}{r + 1} + c$ & Intervalle où $u$ est dérivable et $u^{r}$ est définie \\
\hline
	$\dfrac{u^{\prime}v - uv^{\prime}}{v^2}$ & $\dfrac{u}{v} + c$ & Intervalle où $u$ et $v$ sont dérivables et $v$ ne s'annule pas\\
\hline	
\end{tabular}



















&\\
\hline

\end{tabular}
































\end{document} 