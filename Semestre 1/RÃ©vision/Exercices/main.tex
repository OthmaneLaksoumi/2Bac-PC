\documentclass[12pt,a4paper]{article}
\usepackage[top=2cm,left=1.5cm,right=1cm,bottom=2cm]{geometry}
\usepackage{amssymb,mathtools,amsthm}
\usepackage{fourier}
\usepackage{xcolor}
\usepackage{multicol, array, fancyhdr}
\usepackage{tasks}
\newcommand{\Lim}{\displaystyle\lim}
\renewcommand{\columnseprule}{1pt}
\renewcommand{\arraystretch}{1.5}
% \renewcommand{\frac}[2]{\displaystyle\frac{#1}{#2}}


%======================================================
\newtheoremstyle{mystyle}
{\topsep}% espace avant
{\topsep}% espace après
{\upshape}% police du corps du théorème
{}% indentation (vide pour rien, \parindent)
{\bfseries\sffamily}% police du titre du théorème
{ :}% ponctuation après le théorème
{ }% après le titre du théorème (espace ou \newline)
{%
    \rule[0.5\baselineskip]{0.5\textwidth}{1pt}%
    \newline\fcolorbox{black}{white}{%
    \thmname{#1}\thmnumber{ \textup{#2}}\thmnote{ \textnormal{(#3)}}%
}%
\medskip%
}% spécifications du titre

\theoremstyle{mystyle}
\newtheorem{exo}{Exercice}

%======================================================



\begin{document}


\pagestyle{fancy}
\fancyhf{} % clear all header and footer fields
\fancyhead[L]{Lycée : Zitoun \hspace{1.5cm} Année scolaire : 2024-2025} % Left header
\fancyhead[C]{ \hspace{4cm} Niveau : 2BPC} % Right-Center header
\fancyhead[R]{Prof. Othmane Laksoumi} % Right header
\fancyfoot[C]{\thepage} % Footer


\begin{center}
    \textbf{\LARGE Série de révision}
\end{center}
\begin{multicols*}{2}

\begin{exo}
    Résoudre dans $\mathbb{R}$ les équations suivantes :
    \begin{enumerate}
        \item $2x^2 - 3x - 5 = 0$
        \item $x^2 - 5x + 2 = 0$
        \item $x^4 - 3x^2 + 2 = 0$
        \item $2x^3-13x^2+5x+6 = 0$
        \item $x^2+1 = 0$
    \end{enumerate}
\end{exo}

\begin{exo}
    Déterminer le signe des expressions suivantes :
    \begin{enumerate}
        \item $A(x) = 2x - 3$
        \item $B(x) = -\displaystyle\frac{1}{2}x - 3$
        \item $C(x) = 2x^2 - 3x - 5$
        \item $D(x) = x^2 - 5x + 2$
        \item $E(x) = \displaystyle\frac{x^2-4x-5}{x^2-4}$
    \end{enumerate}
\end{exo}

\begin{exo}
    Déterminer l’ensemble de définition des fonctions suivante
    \begin{enumerate}
        \item $f_1(x) = \displaystyle\frac{1}{x+1} - \displaystyle\frac{1}{x-1}$
        \item $f_2(x) = \sqrt{x - 1}$
        \item $f_3(x) = \displaystyle\frac{x^2-4x+3}{x^2-x-2}$
        \item $f_4(x) = \sqrt{x^2-1}$
        \item $f_5(x) = \displaystyle\frac{\sqrt{x+2}}{x^2+4x+3}$
        \item $f_6(x) = \sqrt{2x^2-3x-5}$
    \end{enumerate}
\end{exo}

\begin{exo}
Calculer les limites suivantes :
\begin{enumerate}
    \item $\Lim_{x\to3}x^2-x+7$
    \item $\Lim_{x\to-\infty}2x^5-7x^4+x^2+1$
    \item $\Lim_{x\rightarrow +\infty} \ \ \frac{2x^{2} +x-2}{-x^{2} -x+6}$
    \item $\Lim_{x\to+\infty}(-3x^3+1)^4(2x-5)$
    \item $\Lim_{x\to-\infty}\frac{x-1}{|4-2x|}$
    \item $\Lim_{x\to+\infty}2x-|4-x|$
    \item $\Lim_{x\to-\infty}2x-|4-x|$
\end{enumerate}
\end{exo}

\begin{exo}
Soit $f$ la fonction numérique définie sur par:
$$(\forall x\in\mathbb{R});\ f(x)=\frac{x^2+\cos{x}}{1+x^2}$$
\begin{enumerate}
    \item Montrer que: $(\forall x\in\mathbb{R});\ \displaystyle\frac{x^2-1}{x^2+1}\leq f(x) \leq 1$
    \item Déduire la limite suivante: $\Lim_{x\to+\infty}\frac{x^2+\cos{x}}{1+x^2}$
\end{enumerate}
\end{exo}

\begin{exo}
    On considére la fonction numérique définie par : 
    $$\begin{cases}
f( x) =  \frac{x\sqrt{x} -1}{x-1}  &  ; \ \ \ \ x >1\\
f( x) =  \frac{3\sqrt{x^{2} +3} -6}{x-1} & ; \ \ \ \ x< 1
\end{cases}$$
\begin{enumerate}
    \item Calculer $\Lim_{x\to1^+} f(x)$ et $\Lim_{x\to1^-} f(x)$
    \item En déduire  $\Lim_{x\to 1} f(x)$
    \item Calculer $\Lim_{x\to-\infty} f(x)$ et $\Lim_{x\to+\infty} f(x)$
\end{enumerate}
\end{exo}

% \begin{exo}
     % Soit $f$ la fonction numérique définie sur $]-2,2[$ par :
     % $$\begin{cases}
     %     f(x) = \frac{\sqrt{x^2+4} - 2}{x} \ \ \ \ \ \ \text{si} \ \ \ -2<x<0\\
     %     f(0) = 0 \\
     %     f(x) = x^3 - 5x^2 \ \ \ \ \ \ \text{si} \ \ \ 0<x<2
     % \end{cases}$$
     % \begin{enumerate}
     %     \item Calculer $\Lim_{x\to 1^+} f(x)$ et $\Lim_{x\to 1^-} f(x)$.
     %     \item En déduire $\Lim_{x\to 1} f(x)$.
     % \end{enumerate}
% \end{exo}

\begin{exo}
    Calculer les limites suivantes :
    \begin{tasks}(2)
        \task[$\bullet$] $\Lim_{x\to +\infty}\displaystyle\frac{\sqrt{2}x^2 + x+ 1}{2x^2-1}$
        \task[$\bullet$] $\Lim_{x\to -\infty}\displaystyle\frac{x^5 + x+ 1}{x^2+x+1}$
        \task[$\bullet$] $\Lim_{x\to +\infty}\displaystyle\frac{\sqrt{x^2 + 2x + 5}}{x}$
        \task[$\bullet$] $\Lim_{x\to 0}\displaystyle\frac{\sin{3x}}{4x}$
        \task[$\bullet$] $\Lim_{x\to 0}\displaystyle\frac{\tan{x}}{\sin{x}}$
        \task[$\bullet$] $\Lim_{x\to 0}\displaystyle\frac{\tan{x}}{\sin{x}}$
        \task[$\bullet$] $\Lim_{x\to 0}\displaystyle\frac{\tan{x}-\sin{x}}{\sqrt{x}}$
        \task[$\bullet$] $\Lim_{x\to 0}\displaystyle\frac{\sin{x}.\sin{2x}}{1-\cos{x}}$
        \task[$\bullet$] $\Lim_{x\to 0}(\sqrt{x^4-x^3} - x^2)$
        \task[$\bullet$] $\Lim_{x\to 0}\displaystyle\frac{x}{\sqrt{x-2}} - \displaystyle\frac{x}{\sqrt{x+2}}$
        \task[$\bullet$] $\Lim_{x\to 4}\displaystyle\frac{\sqrt{2x+1} - 3}{x-4}$
        \task[$\bullet$] $\Lim_{x\to 1^+}\displaystyle\frac{\sqrt{x+3} - \sqrt{3x+1}}{\sqrt{x-1}}$
        
        
        
        
    \end{tasks}
\end{exo}

\end{multicols*}



\end{document}
 